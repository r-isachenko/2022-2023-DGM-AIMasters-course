\input{../utils/preamble}
\createdgmtitle{14}

\usepackage{tikz}

\usetikzlibrary{arrows,shapes,positioning,shadows,trees}
%--------------------------------------------------------------------------------
\begin{document}
%--------------------------------------------------------------------------------
\begin{frame}[noframenumbering,plain]
%\thispagestyle{empty}
\titlepage
\end{frame}
%=======
\begin{frame}{Recap of previous lecture}
	\vspace{-0.5cm}
	\begin{block}{Continuous normalizing flows}
		\[
			\frac{d \log p(\bz(t), t)}{d t} = - \text{tr} \left( \frac{\partial f (\bz(t), \btheta)}{\partial \bz(t)} \right).
		\]
		\vspace{-0.5cm}
	\end{block}
	\begin{block}{Forward transform + log-density}
		\vspace{-0.4cm}
		\[
			\begin{bmatrix}
				\bx \\
				\log p(\bx | \btheta)
			\end{bmatrix}
			= 
			\begin{bmatrix}
				\bz \\
				\log p(\bz)
			\end{bmatrix} + 
			\int_{t_0}^{t_1} 
			\begin{bmatrix}
				f(\bz(t), \btheta) \\
				- \text{tr} \left( \frac{\partial f(\bz(t), \btheta)}{\partial \bz(t)} \right) 
			\end{bmatrix} dt.
		\]
		\vspace{-0.4cm}
	\end{block}	
	\begin{block}{Hutchinson's trace estimator}
		\vspace{-0.8cm}
		\begin{multline*}
		   \log p(\bz(t_1)) = \log p(\bz(t_0)) - \int_{t_0}^{t_1} \text{tr}  \left( \frac{\partial f (\bz(t), \btheta)}{\partial \bz(t)} \right) dt = \\ = \log p(\bz(t_0)) - \mathbb{E}_{p(\bepsilon)} \int_{t_0}^{t_1} \left[ {\color{violet}\bepsilon^T \frac{\partial f}{\partial \bz}} \bepsilon \right] dt.
		\end{multline*}
	\end{block}
	\myfootnotewithlink{https://arxiv.org/abs/1810.01367}{Grathwohl W. et al. FFJORD: Free-form Continuous Dynamics for Scalable Reversible Generative Models, 2018} 
\end{frame}
%=======
\begin{frame}{Recap of previous lecture}
	\vspace{-0.2cm}
	\begin{block}{SDE basics}
		Let define stochastic process $\bx(t)$ with initial condition $\bx(0) \sim p_0(\bx)$:
		\[
			d\bx = \mathbf{f}(\bx, t) dt + g(t) d \bw, 
		\]
		where $\bw(t)$ is the standard Wiener process (Brownian motion)
		\vspace{-0.2cm}
		\[		
			\bw(t) - \bw(s) \sim \cN(0, (t - s) \bI), \quad d \bw = \bepsilon \cdot \sqrt{dt}, \, \text{where } \bepsilon \sim \cN(0, \bI).
		\]
	\end{block}
	\vspace{-0.5cm}
	\begin{block}{Langevin dynamics}
		Let $\bx_0$ be a random vector. Then under mild regularity conditions for small enough $\eta$ samples from the following dynamics
		\vspace{-0.2cm}
		\[
			\bx_{t + 1} = \bx_t + \eta \frac{1}{2} \nabla_{\bx_t} \log p(\bx_t | \btheta) + \sqrt{\eta} \cdot \bepsilon, \quad \bepsilon \sim \cN(0, \bI).
		\]
		will comes from $p(\bx | \btheta)$.
	\end{block}
	The density $p(\bx | \btheta)$ is a \textbf{stationary} distribution for the Langevin SDE.
	\myfootnotewithlink{https://www.stats.ox.ac.uk/~teh/research/compstats/WelTeh2011a.pdf}{Welling M. Bayesian Learning via Stochastic Gradient Langevin Dynamics, 2011}
\end{frame}
%=======
\begin{frame}{Outline}
	\tableofcontents
\end{frame}
%=======
\section{Noise conditioned score network}
%=======
\begin{frame}{Noise conditioned score network}
	\begin{itemize}
		\item Define the sequence of noise levels: $\sigma_1 > \sigma_2 > \dots > \sigma_L$.
		\item Perturb the original data with the different noise level to get $\pi(\bx | \sigma_1), \dots, \pi(\bx | \sigma_L)$.
		\item Train denoised score function $\bs(\bx, \btheta, \sigma)$ for each noise level:
		\vspace{-0.2cm}
		\[
			\sum_{l=1}^L {\color{violet}\sigma_l^2} \bbE_{\pi(\bx')} \bbE_{p(\bx | \bx', \sigma_l)}\bigl\| \bs(\bx, \btheta, \sigma_l) - \nabla_\bx \log p(\bx | \bx', \sigma_l) \bigr\|^2_2 \rightarrow \min_{\btheta}
		\]
		\item Sample from \textbf{annealed} Langevin dynamics (for $l=1, \dots, L$).
	\end{itemize}
	\begin{figure}
		\includegraphics[width=0.6\linewidth]{figs/multi_scale}
	\end{figure}
	\begin{figure}
		\includegraphics[width=\linewidth]{figs/duoduo}
	\end{figure}
	\myfootnotewithlink{https://arxiv.org/abs/1907.05600}{Song Y. et al. Generative Modeling by Estimating Gradients of the Data Distribution, 2019}
\end{frame}
%=======
\begin{frame}{Noise conditioned score network}
	\begin{block}{Samples}
		\begin{figure}
			\includegraphics[width=\linewidth]{figs/NCSNv2}
		\end{figure}
	\end{block}
	\myfootnotewithlink{https://arxiv.org/abs/2006.09011}{Song Y. et al. Improved Techniques for Training Score-Based Generative Models, 2020}
\end{frame}
%=======
\section{Diffusion models}
%=======
\begin{frame}{Forward diffusion process}
	Let $\bx_0 = \bx \sim \pi(\bx)$, $\beta \in (0, 1)$. Define the Markov chain
	\[
		\bx_t = \sqrt{1 - \beta} \bx_{t - 1} + \sqrt{\beta} \bepsilon, \quad \text{where }\bepsilon \sim \cN(0, 1);
	\]
	\[
		q(\bx_t | \bx_{t-1}) = \cN(\bx_t | \sqrt{1 - \beta} \bx_{t-1}, \beta \bI).
	\]
	\vspace{-0.5cm}
	\begin{block}{Statement}
		Applying the Markov chain to samples from any $\pi(\bx)$ we will get $\bx_{\infty} \sim p_{\infty}(\bx) = \cN(0, 1)$. Here $p_{\infty}(\bx)$ is a \textbf{stationary} distribution:
		\vspace{-0.2cm}
		\[
			p_{\infty}(\bx) = \int q(\bx | \bx') p_{\infty}(\bx') d \bx'. 
		\]
		\vspace{-0.8cm}
	\end{block}
	\begin{block}{Statement}
		Denote $\alpha_t = 1 - \beta_t$, $\bar{\alpha}_t = \prod_{s=1}^t \alpha_t$. Then 
		\vspace{-0.2cm}
		\[
			\bx_t = \sqrt{\bar{\alpha}_t} \bx_{0} + \sqrt{1 - \bar{\alpha}_t} \bepsilon, \quad \text{where } \bepsilon \sim \cN(0, 1)
		\]
		\[
			q(\bx_t | \bx_0) = \cN(\bx_t | \sqrt{\bar{\alpha}_t} \bx_0, (1 - \bar{\alpha}_t) \bI).
		\]
		We could sample from any timestamp using only $\bx_0$.
	\end{block}
	\myfootnotewithlink{http://proceedings.mlr.press/v37/sohl-dickstein15.pdf}{Sohl-Dickstein J. Deep Unsupervised Learning using Nonequilibrium Thermodynamics, 2015}
\end{frame}
%=======
\begin{frame}{Forward diffusion process}
	\textbf{Diffusion} refers to the flow of particles from high-density regions towards low-density regions.
	\begin{figure}
		\includegraphics[width=0.5\linewidth]{figs/diffusion_over_time}
	\end{figure}
	\vspace{-0.3cm}
	\begin{enumerate}
		\item $\bx_0 = \bx \sim \pi(\bx)$;
		\item $\bx_t = \sqrt{1 - \beta} \bx_{t - 1} + \sqrt{\beta} \bepsilon$, where $\bepsilon \sim \cN(0, 1)$, $t \geq 1$;
		\item $\bx_T \sim p_{\infty}(\bx) = \cN(0, 1)$.
	\end{enumerate}
	Now our goal is to revert this process.
	\myfootnotewithlink{https://ayandas.me/blog-tut/2021/12/04/diffusion-prob-models.html}{Das A. An introduction to Diffusion Probabilistic Models, blog post, 2021}
\end{frame}
%=======
\begin{frame}{Reverse diffusion process}
	\begin{figure}
		\includegraphics[width=0.8\linewidth]{figs/DDPM}
	\end{figure}
	Let define the reverse process
	\vspace{-0.2cm}
	\[
		p(\bx_{t - 1} | \bx_t, \btheta) = \cN(\bx_{t - 1} | \bmu(\bx_t, \btheta, t), \bsigma^2(\bx_t, \btheta, t))
	\]
	\vspace{-0.5cm}
	\begin{minipage}{0.5\linewidth}
		\begin{block}{Forward process}
			\begin{enumerate}
				\item $\bx_0 = \bx \sim \pi(\bx)$;
				\item $\bx_t = \sqrt{1 - \beta} \bx_{t - 1} + \sqrt{\beta} \bepsilon$, where $\bepsilon \sim \cN(0, 1)$, $t \geq 1$;
				\item $\bx_T \sim p_{\infty}(\bx) = \cN(0, 1)$.
			\end{enumerate}
		\end{block}
	\end{minipage}%
	\begin{minipage}{0.5\linewidth}
		\begin{block}{Reverse process}
			\begin{enumerate}
				\item $\bx_T \sim p_{\infty}(\bx) = \cN(0, 1)$;
				\item $\bx_{t - 1} = \bsigma(\bx_t, \btheta, t) \cdot \bx_t + \bmu(\bx_t, \btheta, t)$;
				\item $\bx_0 = \bx \sim \pi(\bx)$;
			\end{enumerate}
		\end{block}
	\end{minipage}
	\myfootnotewithlink{https://lilianweng.github.io/posts/2021-07-11-diffusion-models/}{Weng L. What are Diffusion Models?, blog post, 2021}
\end{frame}
%=======
\begin{frame}{Diffusion model}
	\begin{figure}
		\includegraphics[width=0.75\linewidth]{figs/diffusion_pgm}
	\end{figure}
	\begin{itemize}
		\item Let treat $\bz = (\bx_1, \dots, \bx_T)$ as a latent variable.
		\item Variational posterior distribution
		\vspace{-0.3cm}
		\[
			q(\bz | \bx) = q(\bx_1, \dots, \bx_T | \bx_0) = \prod_{t = 1}^T q(\bx_t | \bx_{t - 1}).
		\]
		\vspace{-0.5cm}
		\item Probabilistic model
		\[
			p(\bx, \bz | \btheta) = p(\bx | \bz, \btheta) p(\bz | \btheta)
		\]
		\item Generative distribution and prior
		\vspace{-0.3cm}
		\[
			p(\bx | \bz, \btheta) = p(\bx_0 | \bx_1, \btheta); \quad 
			p(\bz | \btheta) = \prod_{t=2}^T p(\bx_{t - 1} | \bx_t, \btheta)
		\]
	\end{itemize}
	\myfootnotewithlink{https://ayandas.me/blog-tut/2021/12/04/diffusion-prob-models.html}{Das A. An introduction to Diffusion Probabilistic Models, blog post, 2021}
\end{frame}
%=======
\begin{frame}{Diffusion model}
	\begin{block}{ELBO}
		\vspace{-0.4cm}
		\[
			\log p(\bx | \btheta) \geq \int q(\bz | \bx) \frac{p(\bx, \bz | \btheta)}{q(\bz | \bx)} d \bz = \cL(q, \btheta) \rightarrow \max_{q, \theta}
		\]
		\vspace{-0.5cm}
	\end{block}
	\begin{block}{Statement}
		\vspace{-0.8cm}
		\begin{multline*}
			\cL(q, \btheta) = \bbE_{q(\bx_1, \dots, \bx_T | \bx_0)}\frac{p(\bx_0, \bx_1, \dots, \bx_T | \btheta)}{q(\bx_1, \dots, \bx_T | \bx_0)} = \\ 
			= \bbE_{q} \Bigl[ {\color{violet}KL\bigl(q(\bx_T | \bx_0) || p(\bx_T)\bigr)}
			+ \sum_{t=2}^T \underbrace{\color{teal}KL \bigl(q(\bx_{t-1} | \bx_t, \bx_0) || p(\bx_{t - 1} || \bx_t, \btheta )\bigr)}_{\cL_t} - \\
			- {\color{olive}\log p(\bx_0 | \bx_1, \btheta)} \Bigr]
		\end{multline*}
		\vspace{-0.5cm}
	\end{block}
	\begin{itemize}
		\item {\color{violet}First term} is constant (KL between two standard normals).
		\item {\color{olive}Third term} is a decoder distribution (could be AR model or discretized distribution (like mixture of logistics)). 
	\end{itemize}
	\myfootnotewithlink{https://arxiv.org/abs/2006.11239}{Ho J. Denoising Diffusion Probabilistic Models, 2020}
\end{frame}
%=======
\begin{frame}{Diffusion model}
	\[
		\cL_t = {\color{teal}KL \bigl(q(\bx_{t-1} | \bx_t, \bx_0) || p(\bx_{t - 1} || \bx_t, \btheta )\bigr)},
	\]
	Here
	\[
		q(\bx_{t-1} | \bx_t, \bx_0) = \cN(\bx_{t-1} | \tilde{\bmu}_t(\bx_t, \bx_0), \tilde{\beta}_t \bI),
	\]
	$\tilde{\bmu}_t(\bx_t, \bx_0)$ and $\tilde{\beta}_t$ have analytical formulas (we omit it) and both dependent on $\beta_t$.
	\begin{itemize}
		\item Assume $\bsigma^2(\bx_t, \btheta, t) = \tilde{\beta}_t \bI$ {\color{gray}(reminder: $p(\bx_{t - 1} | \bx_t, \btheta) = \cN(\bx_{t - 1} | \bmu(\bx_t, \btheta, t), \bsigma^2(\bx_t, \btheta, t))$)}.
		\item Use KL formula for normal distributions.
		\item Use the fact $\bx_t = \sqrt{\bar{\alpha}_t} \bx_{0} + \sqrt{1 - \bar{\alpha}_t} \bepsilon$
	\end{itemize}
	\begin{multline*}
		\cL_t = \bbE_{\bepsilon} \left[ \frac{1}{2\tilde{\beta}_t} \| {\color{violet} \tilde{\bmu}_t(\bx_t, \bx_0)} - \bmu(\bx_t, \btheta, t) \|^2 \right] = \\ 
		= \bbE_{\bepsilon} \left[ \frac{1}{2\tilde{\beta}_t} \left\| {\color{violet} \frac{1}{\sqrt{\alpha_t}}\left( \bx_t - \frac{\beta_t}{\sqrt{1 - \bar{\alpha}_t} } \bepsilon \right)} - \bmu(\bx_t, \btheta, t) \right\|^2 \right]
	\end{multline*}
	\myfootnotewithlink{https://arxiv.org/abs/2006.11239}{Ho J. Denoising Diffusion Probabilistic Models, 2020}
	\end{frame}
%=======
\begin{frame}{Diffusion model}
	\begin{block}{Reparametrization}
		\vspace{-0.3cm}
		\[
			\bmu(\bx_t, \btheta, t) = \frac{1}{\sqrt{\alpha_t}}\left( \bx_t - \frac{\beta_t}{\sqrt{1 - \bar{\alpha}_t} } \bepsilon(\bx_t, \btheta, t) \right) 
		\]
		\vspace{-0.5cm}
	\end{block}
	\begin{block}{KL term}
		\vspace{-0.7cm}
		\begin{multline*}
			\cL_t = \bbE_{\bepsilon} \left[ \frac{1}{2\tilde{\beta}_t} \left\|\frac{1}{\sqrt{\alpha_t}}\left( \bx_t - \frac{\beta_t}{\sqrt{1 - \bar{\alpha}_t} } \bepsilon \right) - \bmu(\bx_t, \btheta, t) \right\|^2 \right] \\ 
			= {\color{olive}\bbE_{\bepsilon}} \left[ \frac{\beta_t^2}{2\tilde{\beta}_t \alpha_t (1 - \bar{\alpha_t})} \left\| {\color{violet}\bepsilon} - {\color{teal}\bepsilon(\bx_t, \btheta, t)}\right\|^2 \right]
		\end{multline*}
		\vspace{-0.5cm}
	\end{block}
	\begin{block}{Noise conditioned score network}
		\vspace{-0.2cm}
		\[
			{\color{olive}\bbE_{p(\bx | \bx', \sigma_l)}}\bigl\| {\color{teal}\bs(\bx, \btheta, \sigma_l)} - {\color{violet}\nabla_\bx \log p(\bx | \bx', \sigma_l)} \bigr\|^2_2 \rightarrow \min_{\btheta}
		\]
	\end{block}
	\myfootnotewithlink{https://arxiv.org/abs/2006.11239}{Ho J. Denoising Diffusion Probabilistic Models, 2020}
	\end{frame}
%=======
\begin{frame}{Denoising diffusion probabilistic model}
	\begin{block}{Samples}
		\begin{figure}
			\includegraphics[width=\linewidth]{figs/ddpm_samples}
		\end{figure}
	\end{block}
	\myfootnotewithlink{https://arxiv.org/abs/2006.11239}{Ho J. Denoising Diffusion Probabilistic Models, 2020}
\end{frame}
%=======
\begin{frame}{The poorest course overview :)}
	\begin{figure}
		\includegraphics[width=\linewidth]{figs/generative-overview}
	\end{figure}
	\myfootnotewithlink{https://lilianweng.github.io/posts/2021-07-11-diffusion-models/}{Weng L. What are Diffusion Models?, blog post, 2021}
\end{frame}
%=======
\begin{frame}{Summary}
	\begin{itemize}
		\item Noise conditioned score network uses multiple noise levels and annealed Langevin dynamics to fit score function.
		\vfill
		\item Gaussian diffusion process is a Markov chain that inject Gaussian noise.
		\vfill
		\item Diffusion model is a VAE model which revert gaussian diffusion process using variational inference.
		\vfill
		\item Objective of diffusion model is closely related to the noise conditioned score network.
	\end{itemize}
\end{frame}
\end{document} 